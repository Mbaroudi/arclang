\documentclass[11pt,a4paper]{article}

% Packages
\usepackage[utf8]{inputenc}
\usepackage[T1]{fontenc}
\usepackage[english]{babel}
\usepackage{geometry}
\usepackage{graphicx}
\usepackage{xcolor}
\usepackage{listings}
\usepackage{fancyhdr}
\usepackage{hyperref}
\usepackage{tcolorbox}
\usepackage{tikz}
\usepackage{svg}
\usepackage{longtable}
\usepackage{booktabs}
\usepackage{enumitem}
\usepackage{pdflscape}  % For landscape pages

% Page geometry - A4 portrait default, A3 landscape for diagram
\geometry{
    a4paper,
    left=25mm,
    right=25mm,
    top=30mm,
    bottom=30mm
}

% Colors
\definecolor{arcblue}{RGB}{26,35,126}
\definecolor{arcgreen}{RGB}{76,175,80}
\definecolor{arcorange}{RGB}{255,152,0}
\definecolor{codebg}{RGB}{245,245,245}
\definecolor{commentgreen}{RGB}{34,139,34}

% Code listing style for ArcLang
\lstdefinelanguage{ArcLang}{
    keywords={model, metadata, requirements, architecture, component, interface, function, trace, satisfies, req, interface_in, interface_out, stakeholder, system, functional, safety, security, regulatory, logical, physical},
    keywordstyle=\color{arcblue}\bfseries,
    ndkeywords={version, description, domain, safety_standard, security_standard, project_phase, priority, safety_level, layer, stereotype, protocol, format, from, to, id},
    ndkeywordstyle=\color{arcorange},
    sensitive=true,
    comment=[l]{//},
    commentstyle=\color{commentgreen}\itshape,
    string=[s]{"}{"},
    stringstyle=\color{red},
    basicstyle=\ttfamily\footnotesize,
    breaklines=true,
    numbers=left,
    numberstyle=\tiny\color{gray},
    numbersep=5pt,
    frame=single,
    backgroundcolor=\color{codebg},
    captionpos=b,
    tabsize=2,
    showstringspaces=false
}

% Headers and footers
\pagestyle{fancy}
\fancyhf{}
\fancyhead[L]{\leftmark}
\fancyhead[R]{\thepage}
\fancyfoot[C]{Vehicle Remote Start System - Arcadia Architecture}
\renewcommand{\headrulewidth}{0.4pt}
\renewcommand{\footrulewidth}{0.4pt}

% Hyperref setup
\hypersetup{
    colorlinks=true,
    linkcolor=arcblue,
    filecolor=arcblue,
    urlcolor=arcblue,
    citecolor=arcblue,
    pdftitle={Vehicle Remote Start System Architecture},
    pdfauthor={ArcLang Compiler},
    pdfsubject={Arcadia MBSE Architecture},
    pdfkeywords={Arcadia, MBSE, Automotive, ISO 26262, ArcLang}
}

% Title page
\title{
    \vspace{-2cm}
    \includegraphics[width=3cm]{example-image} \\[1cm]
    {\Huge\bfseries Vehicle Remote Start System}\\[0.5cm]
    {\Large Arcadia Architecture with ArcLang DSL}\\[0.3cm]
    {\large ISO 26262 ASIL B Compliant}
}
\author{
    Generated with ArcLang Compiler\\
    \texttt{https://github.com/arclang}
}
\date{\today}

\begin{document}

\maketitle
\thispagestyle{empty}

\begin{abstract}
This document presents a complete Arcadia Model-Based Systems Engineering (MBSE) architecture for a Vehicle Remote Start System implemented using the ArcLang Domain-Specific Language (DSL). The system enables users to remotely start their vehicles via smartphone while ensuring ISO 26262 ASIL B safety compliance and ISO/SAE 21434 cybersecurity standards. The architecture covers all Arcadia layers: Operational Analysis, System Analysis, Logical Architecture, and Physical Architecture, with complete traceability between stakeholder requirements, system requirements, and implementation components.
\end{abstract}

\newpage
\tableofcontents
\newpage

% ============================================================================
\section{Introduction}
% ============================================================================

\subsection{Purpose}
This document describes the complete architecture of a Vehicle Remote Start System designed using the Arcadia methodology and implemented with the ArcLang Domain-Specific Language. The system allows vehicle owners to remotely start their vehicles from a smartphone application while maintaining strict safety and security requirements.

\subsection{Scope}
The architecture includes:
\begin{itemize}[leftmargin=*]
    \item \textbf{25 Logical Components} organized in 4 architectural layers
    \item \textbf{33 Requirements} spanning stakeholder, system, functional, safety, security, and regulatory domains
    \item \textbf{16 Interfaces} defining component interactions
    \item \textbf{32 Functions} implementing system behaviors
    \item \textbf{Complete traceability} from stakeholder needs to physical implementation
\end{itemize}

\subsection{Standards Compliance}
\begin{tcolorbox}[colback=arcblue!5,colframe=arcblue,title=Safety \& Security Standards]
\begin{itemize}[leftmargin=*]
    \item \textbf{ISO 26262:2018} - Functional Safety (ASIL B for remote start controller)
    \item \textbf{ISO/SAE 21434:2021} - Cybersecurity Engineering
    \item \textbf{UNECE R100} - Electric Vehicle Safety
    \item \textbf{FCC Part 15} - Radiated Emissions
    \item \textbf{GDPR} - Data Privacy and Protection
\end{itemize}
\end{tcolorbox}

\subsection{Arcadia Methodology}
Arcadia (Architecture Analysis and Design Integrated Approach) is a model-based methodology that structures system architecture into four main layers:

\begin{enumerate}[leftmargin=*]
    \item \textbf{Operational Analysis} - Understand stakeholder needs and use cases
    \item \textbf{System Analysis} - Define system requirements and behaviors
    \item \textbf{Logical Architecture} - Design logical components and interactions
    \item \textbf{Physical Architecture} - Allocate to hardware/software elements
\end{enumerate}

% ============================================================================
\section{System Overview}
% ============================================================================

\subsection{System Context}
The Vehicle Remote Start System enables users to start their vehicle remotely via a smartphone application. The system must:
\begin{itemize}[leftmargin=*]
    \item Authenticate users securely using cryptographic tokens
    \item Validate all safety preconditions (parking brake, neutral, doors closed)
    \item Support multiple powertrain types (ICE, Hybrid, Electric)
    \item Provide climate pre-conditioning for cabin comfort
    \item Automatically shut down after 10 minutes
    \item Maintain complete audit trail for regulatory compliance
\end{itemize}

\subsection{Key Features}
\begin{tcolorbox}[colback=arcgreen!5,colframe=arcgreen,title=System Capabilities]
\begin{itemize}[leftmargin=*]
    \item \textbf{Multi-Powertrain Support} - ICE, Hybrid, and EV start sequences
    \item \textbf{Safety Interlocks} - 200ms validation of parking brake, transmission, doors
    \item \textbf{Secure Communication} - AES-256 encryption with certificate authentication
    \item \textbf{Climate Control} - Automatic HVAC pre-conditioning
    \item \textbf{Auto-Shutdown} - 10-minute timer with manual override
    \item \textbf{Replay Protection} - Cryptographic nonce with 5-second window
    \item \textbf{Audit Logging} - Complete trail for regulatory compliance
\end{itemize}
\end{tcolorbox}

% ============================================================================
\section{ArcLang Source Code}
% ============================================================================

\subsection{About ArcLang}
ArcLang is a Domain-Specific Language designed for expressing Arcadia architectures in a concise, readable, and maintainable format. Key features include:
\begin{itemize}[leftmargin=*]
    \item \textbf{Declarative Syntax} - Express architecture intent clearly
    \item \textbf{Built-in Traceability} - Automatic requirement tracing
    \item \textbf{Safety Annotations} - Native ASIL level support
    \item \textbf{Multi-Format Export} - Generate diagrams, documents, code
    \item \textbf{Validation Engine} - Catch architectural inconsistencies early
\end{itemize}

\subsection{Complete Source Code}

The complete ArcLang source code for the Vehicle Remote Start System is shown below. This single file defines the entire architecture including requirements, components, interfaces, and traceability.

\lstinputlisting[
    language=ArcLang,
    caption={Remote Start System Architecture (remote\_start\_architecture.arc)},
    label={lst:arclang},
    basicstyle=\ttfamily\scriptsize,
    breaklines=true
]{remote_start_architecture.arc}

% ============================================================================
\section{Architecture Diagrams}
% ============================================================================

\subsection{Logical Architecture Diagram}

Figure~\ref{fig:logical_arch} shows the complete logical architecture with 25 components organized across 4 layers (User, Connectivity, Control, Vehicle). The diagram follows Capella visualization standards with:

\begin{itemize}[leftmargin=*]
    \item \textbf{Port Distribution} - Required interfaces (IN) on left (green), Provided interfaces (OUT) on right (orange)
    \item \textbf{Layer Swimlanes} - Dashed borders separating architectural layers
    \item \textbf{ASIL Badges} - Safety-critical components marked with ASIL level (B, C, D)
    \item \textbf{Exchange Items} - Interface protocols labeled on connections (CAN, LIN, HTTPS)
    \item \textbf{Stereotype Icons} - Visual indicators for component types (Gateway, Controller, ECU)
\end{itemize}

% Full page landscape A3 diagram
\begin{landscape}
\begin{figure}[p]
    \centering
    \includesvg[width=0.95\textheight,keepaspectratio]{remote_start_diagram.svg}
    \caption{Logical Architecture - Vehicle Remote Start System (Capella Standard)}
    \label{fig:logical_arch}
\end{figure}
\end{landscape}

\textbf{Diagram Generation Command:}
\begin{tcolorbox}[colback=codebg,colframe=black]
\texttt{\$ cd /Users/malek/Arclang}\\
\texttt{\$ cargo run --bin arclang -- explorer examples/remote\_start\_architecture.arc}\\
\texttt{\$ open examples/remote\_start\_architecture\_explorer.html}\\
\texttt{\$ \# Click "Export SVG" button to generate diagram file}
\end{tcolorbox}

\subsection{Architecture Layers}

\subsubsection{User Layer (LA-USER)}
\begin{itemize}[leftmargin=*]
    \item \textbf{Smartphone Application} - Mobile UI for remote vehicle control
    \item \textbf{User Authentication Service} - Cryptographic token generation and biometric validation
\end{itemize}

\subsubsection{Connectivity Layer (LA-CONN)}
\begin{itemize}[leftmargin=*]
    \item \textbf{Telematics Control Unit} - LTE/5G cellular gateway (ASIL B)
    \item \textbf{Secure Communication Manager} - Cryptographic signature verification and replay attack detection
    \item \textbf{Cloud Backend Service} - Request queuing and audit trail storage
\end{itemize}

\subsubsection{Control Layer (LA-CTRL)}
\begin{itemize}[leftmargin=*]
    \item \textbf{Remote Start Controller} - Orchestrates start sequence (ASIL B)
    \item \textbf{Safety Interlock Validator} - Validates safety preconditions (ASIL B)
    \item \textbf{Powertrain Start Manager} - Multi-powertrain start sequences (ASIL B)
    \item \textbf{Climate Control Pre-Conditioner} - HVAC activation
    \item \textbf{Timer and Shutdown Manager} - 10-minute auto-shutdown (ASIL B)
\end{itemize}

\subsubsection{Vehicle Layer (LA-VHC)}
\begin{itemize}[leftmargin=*]
    \item \textbf{Engine Control Unit} - Fuel injection and ignition (ASIL B)
    \item \textbf{Battery Management System} - HV battery monitoring (ASIL B)
    \item \textbf{Hybrid Control Unit} - Engine/motor coordination (ASIL B)
    \item \textbf{HVAC Control Module} - Climate control
    \item \textbf{Body Control Module} - Door/brake sensors
    \item \textbf{Instrument Cluster} - Status display
\end{itemize}

% ============================================================================
\section{Requirements Breakdown}
% ============================================================================

\subsection{Stakeholder Requirements (8 total)}

\begin{longtable}{|p{2.5cm}|p{10cm}|p{2cm}|}
\caption{Stakeholder Requirements}\label{tab:stakeholder_req}\\
\hline
\textbf{ID} & \textbf{Description} & \textbf{Priority} \\
\hline
\endfirsthead
\hline
\textbf{ID} & \textbf{Description} & \textbf{Priority} \\
\hline
\endhead
\hline
\endfoot

STK-RS-001 & User must be able to remotely start the vehicle from their smartphone within 100m range & Critical \\
\hline
STK-RS-002 & System must prevent remote start if vehicle safety conditions are not met & Critical (ASIL B) \\
\hline
STK-RS-003 & Remote start must work for ICE, Hybrid, and Electric vehicles with appropriate powertrain control & Critical \\
\hline
STK-RS-004 & System must authenticate user and secure all communications to prevent unauthorized access & Critical \\
\hline
STK-RS-005 & Vehicle must comply with regional emission and safety regulations during remote start & High \\
\hline
STK-RS-006 & User shall receive confirmation of remote start status within 3 seconds & High \\
\hline
STK-RS-007 & System must automatically shut down after 10 minutes if not manually overridden & Medium \\
\hline
STK-RS-008 & Remote start must maintain cabin comfort through climate control pre-conditioning & Medium \\
\hline
\end{longtable}

\subsection{System Requirements (10 total)}

\begin{longtable}{|p{2.5cm}|p{10cm}|p{2cm}|}
\caption{System Requirements}\label{tab:system_req}\\
\hline
\textbf{ID} & \textbf{Description} & \textbf{Priority} \\
\hline
\endfirsthead
\hline
\textbf{ID} & \textbf{Description} & \textbf{Priority} \\
\hline
\endhead
\hline
\endfoot

SYS-RS-001 & System shall authenticate user identity using cryptographic tokens before allowing remote start & Critical (ASIL B) \\
\hline
SYS-RS-002 & System shall verify all safety interlocks (parking brake, neutral, doors closed) before starting & Critical (ASIL B) \\
\hline
SYS-RS-003 & System shall establish secure encrypted communication channel between smartphone and vehicle & Critical \\
\hline
SYS-RS-004 & System shall monitor battery state and prevent start if below 20\% (EV) or 11V (ICE) & High \\
\hline
SYS-RS-005 & System shall activate appropriate powertrain control strategy based on vehicle type & Critical \\
\hline
SYS-RS-006 & System shall send status notifications (success/failure/warnings) to user smartphone & High \\
\hline
SYS-RS-007 & System shall implement automatic shutdown timer with maximum 10 minute duration & High \\
\hline
SYS-RS-008 & System shall activate climate control to target temperature based on user preferences & Medium \\
\hline
SYS-RS-009 & System shall log all remote start attempts with timestamp and result for audit trail & Medium \\
\hline
SYS-RS-010 & System shall detect tampering or replay attacks and raise security alerts & Critical \\
\hline
\end{longtable}

\subsection{Functional Requirements (4 total)}

\begin{longtable}{|p{2.5cm}|p{10cm}|p{2cm}|}
\caption{Functional Requirements}\label{tab:functional_req}\\
\hline
\textbf{ID} & \textbf{Description} & \textbf{Priority} \\
\hline
\endfirsthead
\hline
\textbf{ID} & \textbf{Description} & \textbf{Priority} \\
\hline
\endhead
\hline
\endfoot

FUNC-RS-001 & Remote start request processing shall validate user credentials within 500ms & Critical \\
\hline
FUNC-RS-002 & Safety interlock validation shall check all preconditions within 200ms & Critical (ASIL B) \\
\hline
FUNC-RS-003 & Powertrain control shall initiate start sequence based on vehicle type & Critical \\
\hline
FUNC-RS-004 & Notification service shall deliver status updates with 99.9\% reliability & High \\
\hline
\end{longtable}

\subsection{Safety Requirements (3 total)}

\begin{longtable}{|p{2.5cm}|p{10cm}|p{2cm}|}
\caption{Safety Requirements (ISO 26262)}\label{tab:safety_req}\\
\hline
\textbf{ID} & \textbf{Description} & \textbf{ASIL} \\
\hline
\endfirsthead
\hline
\textbf{ID} & \textbf{Description} & \textbf{ASIL} \\
\hline
\endhead
\hline
\endfoot

SAFE-RS-001 & Remote start controller shall implement watchdog monitoring with 100ms timeout & ASIL B \\
\hline
SAFE-RS-002 & Safety interlock failure shall inhibit remote start and log fault code & ASIL B \\
\hline
SAFE-RS-003 & Remote start shall be inhibited if any ASIL B or higher fault is active & ASIL B \\
\hline
\end{longtable}

\subsection{Security Requirements (4 total)}

\begin{longtable}{|p{2.5cm}|p{10cm}|p{2cm}|}
\caption{Security Requirements (ISO/SAE 21434)}\label{tab:security_req}\\
\hline
\textbf{ID} & \textbf{Description} & \textbf{Priority} \\
\hline
\endfirsthead
\hline
\textbf{ID} & \textbf{Description} & \textbf{Priority} \\
\hline
\endhead
\hline
\endfoot

SEC-RS-001 & All remote commands shall use AES-256 encryption with perfect forward secrecy & Critical \\
\hline
SEC-RS-002 & System shall implement certificate-based authentication with hardware-backed key storage & Critical \\
\hline
SEC-RS-003 & Replay attack detection shall use cryptographic nonce with maximum 5 second window & Critical \\
\hline
SEC-RS-004 & Failed authentication attempts shall trigger exponential backoff after 3 failures & High \\
\hline
\end{longtable}

\subsection{Regulatory Requirements (4 total)}

\begin{longtable}{|p{2.5cm}|p{10cm}|p{2cm}|}
\caption{Regulatory Compliance Requirements}\label{tab:regulatory_req}\\
\hline
\textbf{ID} & \textbf{Description} & \textbf{Priority} \\
\hline
\endfirsthead
\hline
\textbf{ID} & \textbf{Description} & \textbf{Priority} \\
\hline
\endhead
\hline
\endfoot

REG-RS-001 & System shall comply with FCC Part 15 for radiated emissions & Critical \\
\hline
REG-RS-002 & Remote start duration shall comply with local idle time regulations (max 10 min) & High \\
\hline
REG-RS-003 & System shall comply with GDPR for user data privacy and consent & High \\
\hline
REG-RS-004 & Remote start shall comply with UNECE R100 for electric vehicle safety & High \\
\hline
\end{longtable}

% ============================================================================
\section{Interface Specifications}
% ============================================================================

\subsection{Communication Protocols}

The system uses multiple communication protocols across different layers:

\begin{table}[htbp]
\centering
\caption{Communication Protocols by Layer}
\label{tab:protocols}
\begin{tabular}{|l|l|p{7cm}|}
\hline
\textbf{Layer} & \textbf{Protocol} & \textbf{Usage} \\
\hline
User & HTTPS & Smartphone to cloud backend \\
\hline
User & OAuth/JWT & Authentication token exchange \\
\hline
Connectivity & LTE/5G & Cloud to vehicle TCU \\
\hline
Connectivity & MQTT & Cloud pub/sub messaging \\
\hline
Control & CAN & High-speed vehicle bus (500 Kbps) \\
\hline
Control & CAN FD & Fast CAN for HV systems (2 Mbps) \\
\hline
Vehicle & LIN & Body electronics (19.2 Kbps) \\
\hline
\end{tabular}
\end{table}

\subsection{Key Interface Flows}

\subsubsection{Remote Start Request Flow}
\begin{enumerate}[leftmargin=*]
    \item User initiates remote start via smartphone app
    \item App generates authentication token and encrypts request
    \item HTTPS request sent to cloud backend
    \item Cloud validates token and queues request
    \item MQTT message sent to vehicle TCU
    \item TCU decrypts and validates command signature
    \item CAN message sent to Remote Start Controller
    \item Controller validates safety interlocks
    \item Start command sent to appropriate powertrain (ECM/BMS/HCU)
    \item Status confirmation sent back through chain
\end{enumerate}

\subsubsection{Safety Interlock Validation Flow}
\begin{enumerate}[leftmargin=*]
    \item Remote Start Controller requests interlock validation
    \item Safety Interlock Validator queries Body Control Module
    \item BCM reads door sensors, parking brake, and transmission position
    \item Interlock status returned to controller (200ms max)
    \item If valid, start sequence proceeds
    \item If invalid, request denied and fault logged
\end{enumerate}

% ============================================================================
\section{ArcViz Engine Configuration}
% ============================================================================

\subsection{Capella Diagram Standards}

The architectural diagrams are generated using the ArcViz engine, configured to follow Eclipse Capella visualization standards:

\begin{tcolorbox}[colback=arcblue!5,colframe=arcblue,title=ArcViz Capella Configuration]
\begin{itemize}[leftmargin=*]
    \item \textbf{Port Distribution} - IN (required) left/green, OUT (provided) right/orange
    \item \textbf{Port Spacing} - 50px vertical spacing to prevent label overlap
    \item \textbf{Node Spacing} - 350px horizontal, 200px vertical between components
    \item \textbf{Layer Swimlanes} - Dashed borders (8px dash, 4px gap) with 50px top padding
    \item \textbf{Exchange Items} - White label boxes on edges showing protocols
    \item \textbf{ASIL Badges} - Colored circles (B=orange, C=red, D=dark red)
    \item \textbf{Auto-Sizing} - Dynamic component sizing based on port count and functions
    \item \textbf{Zero Overlaps} - Guaranteed through Dagre layout + clipping
\end{itemize}
\end{tcolorbox}

\subsection{Layout Configuration}

\begin{lstlisting}[language=JavaScript,caption={ArcViz Engine Configuration},basicstyle=\ttfamily\scriptsize]
const ARCVIZ_CONFIG = {
    layout: {
        rankdir: 'TB',        // Top-to-bottom layout
        nodesep: 350,         // Horizontal spacing (px)
        ranksep: 200,         // Vertical spacing (px)
        marginx: 150,         // Left/right margins (px)
        marginy: 100,         // Top/bottom margins (px)
        edgesep: 100          // Edge separation (px)
    },
    port: {
        size: 12,             // Port square size (px)
        spacing: 50,          // Vertical port spacing (px)
        colors: {
            inFill: '#4caf50',   // Green for IN ports
            outFill: '#ff9800'   // Orange for OUT ports
        }
    },
    safety: {
        colors: {
            ASIL_B: '#ff9800',   // Orange
            ASIL_C: '#f44336',   // Red  
            ASIL_D: '#d32f2f'    // Dark red
        }
    }
};
\end{lstlisting}

% ============================================================================
\section{Traceability Matrix}
% ============================================================================

\subsection{Requirements Traceability}

Complete traceability from stakeholder requirements through system requirements to functional requirements:

\begin{table}[htbp]
\centering
\caption{Requirements Traceability Matrix}
\label{tab:traceability}
\begin{tabular}{|l|l|p{7cm}|}
\hline
\textbf{Stakeholder} & \textbf{System} & \textbf{Functional} \\
\hline
STK-RS-001 & SYS-RS-001 & FUNC-RS-001 \\
 & SYS-RS-003 & \\
\hline
STK-RS-002 & SYS-RS-002 & FUNC-RS-002 \\
\hline
STK-RS-003 & SYS-RS-005 & FUNC-RS-003 \\
\hline
STK-RS-004 & SYS-RS-001 & FUNC-RS-001 \\
 & SYS-RS-003 & \\
 & SYS-RS-010 & \\
\hline
STK-RS-006 & SYS-RS-006 & FUNC-RS-004 \\
\hline
STK-RS-007 & SYS-RS-007 & - \\
\hline
STK-RS-008 & SYS-RS-008 & - \\
\hline
\end{tabular}
\end{table}

% ============================================================================
\section{Compilation and Export}
% ============================================================================

\subsection{Compiling ArcLang to Diagrams}

To generate the architecture diagrams and explorer from the ArcLang source:

\begin{tcolorbox}[colback=codebg,colframe=black,title=Compilation Commands]
\begin{verbatim}
# Navigate to ArcLang directory
cd /Users/malek/Arclang

# Compile ArcLang to Architecture Explorer (HTML)
cargo run --bin arclang -- explorer \
    examples/remote_start_architecture.arc

# Open the generated explorer
open examples/remote_start_architecture_explorer.html

# Export diagram as SVG for LaTeX inclusion
# (Click "Export SVG" button in the web interface)
\end{verbatim}
\end{tcolorbox}

\subsection{Export Formats}

The ArcLang compiler supports multiple export formats:

\begin{itemize}[leftmargin=*]
    \item \textbf{HTML Explorer} - Interactive web-based architecture browser
    \item \textbf{SVG} - Scalable vector graphics for documentation
    \item \textbf{PNG} - Raster images for presentations
    \item \textbf{PDF} - High-quality print output
    \item \textbf{JSON} - Machine-readable architecture data
    \item \textbf{Markdown} - Requirements documentation
\end{itemize}

% ============================================================================
\section{Conclusion}
% ============================================================================

\subsection{Summary}

This document has presented a complete Arcadia architecture for a Vehicle Remote Start System, implemented using the ArcLang Domain-Specific Language. The architecture demonstrates:

\begin{itemize}[leftmargin=*]
    \item \textbf{Complete Arcadia Methodology} - All 4 layers from stakeholder needs to physical implementation
    \item \textbf{Safety Compliance} - ISO 26262 ASIL B for safety-critical components
    \item \textbf{Security-by-Design} - ISO/SAE 21434 with AES-256 encryption and certificate authentication
    \item \textbf{Capella Standards} - Professional diagrams following Eclipse Capella conventions
    \item \textbf{Full Traceability} - Every stakeholder requirement traced to implementation
    \item \textbf{DSL Benefits} - Concise, maintainable architecture in 807 lines of ArcLang code
\end{itemize}

\subsection{Architecture Statistics}

\begin{table}[htbp]
\centering
\caption{Architecture Metrics}
\label{tab:metrics}
\begin{tabular}{|l|r|l|}
\hline
\textbf{Metric} & \textbf{Count} & \textbf{Details} \\
\hline
ArcLang Lines of Code & 807 & Complete architecture definition \\
\hline
Requirements & 33 & Stakeholder, System, Functional, Safety, Security, Regulatory \\
\hline
Logical Components & 25 & Across 4 architectural layers \\
\hline
Physical Components & 9 & ECUs and software modules \\
\hline
Interfaces & 16 & Component-to-component connections \\
\hline
Functions & 32 & Behavioral operations \\
\hline
Trace Links & 8 & Requirement traceability \\
\hline
ASIL B Components & 10 & Safety-critical elements \\
\hline
Communication Protocols & 7 & HTTPS, CAN, CAN FD, LIN, MQTT, OAuth, LTE \\
\hline
\end{tabular}
\end{table}

\subsection{Next Steps}

\begin{enumerate}[leftmargin=*]
    \item \textbf{Physical Architecture Detailing} - Complete ECU pinout and network topology
    \item \textbf{Safety Analysis} - FMEA, FTA, FMEDA for ASIL B components
    \item \textbf{Security Threat Analysis} - TARA per ISO/SAE 21434
    \item \textbf{Code Generation} - Generate AUTOSAR RTE configuration
    \item \textbf{V\&V Planning} - Test cases derived from requirements
    \item \textbf{Simulation} - Model-in-the-loop validation
\end{enumerate}

% ============================================================================
\section*{Appendices}
% ============================================================================

\appendix

\section{ArcLang Language Reference}
\label{app:arclang_ref}

ArcLang keywords and constructs used in this architecture:

\begin{description}[leftmargin=3cm,style=nextline]
    \item[\texttt{model}] Top-level architecture model declaration
    \item[\texttt{metadata}] Model metadata (version, standards, domain)
    \item[\texttt{requirements}] Requirement block (stakeholder, system, functional, safety, security, regulatory)
    \item[\texttt{req}] Individual requirement with ID and description
    \item[\texttt{architecture}] Architecture layer (logical, physical)
    \item[\texttt{component}] Architectural component with ID, layer, stereotype
    \item[\texttt{interface\_in}] Required interface (input port)
    \item[\texttt{interface\_out}] Provided interface (output port)
    \item[\texttt{function}] Behavioral function within component
    \item[\texttt{interface}] Connection between components (from/to)
    \item[\texttt{trace}] Traceability link (satisfies)
    \item[\texttt{safety\_level}] ASIL level (ASIL\_B, ASIL\_C, ASIL\_D)
    \item[\texttt{priority}] Requirement priority (Critical, High, Medium, Low)
\end{description}

\section{Abbreviations and Acronyms}
\label{app:acronyms}

\begin{description}[leftmargin=3cm,style=nextline]
    \item[ASIL] Automotive Safety Integrity Level
    \item[BCM] Body Control Module
    \item[BMS] Battery Management System
    \item[CAN] Controller Area Network
    \item[ECU] Electronic Control Unit
    \item[FD] Flexible Data-rate (CAN FD)
    \item[FMEA] Failure Mode and Effects Analysis
    \item[FTA] Fault Tree Analysis
    \item[GDPR] General Data Protection Regulation
    \item[HCU] Hybrid Control Unit
    \item[HVAC] Heating, Ventilation, and Air Conditioning
    \item[HV] High Voltage
    \item[ICE] Internal Combustion Engine
    \item[JWT] JSON Web Token
    \item[LIN] Local Interconnect Network
    \item[MBSE] Model-Based Systems Engineering
    \item[MQTT] Message Queuing Telemetry Transport
    \item[TCU] Telematics Control Unit
    \item[V\&V] Verification and Validation
\end{description}

\section{References}
\label{app:references}

\begin{enumerate}[leftmargin=*]
    \item ISO 26262:2018 - Road vehicles — Functional safety
    \item ISO/SAE 21434:2021 - Road vehicles — Cybersecurity engineering
    \item UNECE Regulation No. 100 - Electric power trained vehicles
    \item Arcadia Methodology - Eclipse Capella project documentation
    \item ArcLang DSL Specification - \url{https://github.com/arclang}
    \item FCC Part 15 - Radio Frequency Devices
    \item GDPR - General Data Protection Regulation (EU 2016/679)
\end{enumerate}

% ============================================================================
% Document end
% ============================================================================

\end{document}
